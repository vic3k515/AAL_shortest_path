\documentclass[11pt,a4paper]{article}

\usepackage[utf8]{inputenc}

\usepackage[T1]{fontenc}
\usepackage{polski}
\usepackage{enumitem}
\usepackage{amsmath}
\usepackage{algorithm}
\usepackage{longtable}
\usepackage{array}
\usepackage[noend]{algpseudocode}
\makeatletter
\def\BState{\State\hskip-\ALG@thistlm}
\makeatother

\title{Najkrótsza droga w mieście}
\author{Wiktor Franus}

\begin{document}
\maketitle

\begin{abstract}
	\begin{center}
	Aplikacja prezentująca działanie algorytmów znajdujących najkrótszą ścieżkę
	pomiędzy dwoma białymi punktami na czarno-białym rastrze.
	\end{center}
\end{abstract}

\tableofcontents
\newpage

\section{Opis problemu}
Dany jest raster MxN o polach białych i czarnych. Opracować algorytm, który znajdzie najkrótszą drogę z
białego pola A do białego pola B, pod warunkiem, że mozna się poruszać jedynie w pionie i w poziomie
omjając przy tym pola czarne. Przy generacji danych należy zwrócić uwagę, aby z każdego pola białego
można było się potencjalnie przedostać do dowolnego innego pola o tym kolorze.\\
Raster można traktować jako plan miasta lub labirynt, w którym białe pola reprezentują drogi, a czarne
budynki lub ściany. Każde białe pole może być punktem początkowym, z którego chcemy znaleźć najkrótszą
drogę do innego wybranego białego pola. Zgodnie z założeniami problemu znalezienie takiej drogi zawsze
jest możliwe. Długość drogi z punktu A do punktu B liczona jest jako ilość pól białych, które należy
odwiedzić, aby przejść z pola A do pola B. Nie można przy tym odwiedzać 2 razy tego samego pola.
Możliwe jest istnienie więcej niż jednej drogi z punktu A do punktu B oraz istnienie kilku dróg o jednakowej
długości. Szukana jest długość najktótszej drogi z A do B wraz z ideksami pól tworzących tę drogę.
Zadany problem można inaczej przedstawić jako problem szukania najkrótszej ścieżki pomiędzy dwoma
wierzchołkami w spójnym, nieskierowanym grafie.
W tak sformułowanym zadaniu wszystkie białe pola rastra można interpretować jako wierzchołki grafu. Jeśli
białe pola sąsiadują ze sobą na rastrze, tj. są położone obok siebie w jednej kolumnie lub w jednym wierszu,
w grafie łączy je krawędź. Długość każdej takiej krawędzi jest równa 1, zatem powstały graf nie jest grafem
ważonym. Spośród wszystkich wierzchołków wyróżnione są 2 wierzchołki: startowy i początkowy.

\section{Metody rozwiązania}
Istnieje wiele algorytmów grafowych, które odnajdują najkrótsze ścieżki pomiędzy
wierzchołkami. W niniejszym projekcie wybrane i zaimplementowane zostały trzy
algorytmy, wykazujące się zarówno prostotą działania, jak i niską złożonością
czasową. Są to: algorytm Breadth-First Search, algorytm Dijkstry oraz heurystyczny algorytm A*.

\subsection{Breadth-First Search}
Algorytm rozpoczyna pracę w wierzchołku początkowym i odwiedza jego wszystkie sąsiednie wierzchołki,
następnie odwiedza sąsiadów każdego sąsiada, itd. W każdym kroku zapamiętuje wskazanie na poprzednika.
W ten sposób, po zakończeniu działania algorytmu możliwe jest odtworzenie
najkrótszej ścieżki. Algorytm kończy się, gdy analizowany wierzchołek jest wierzchołkiem końcowym.

\subsubsection{Pseudokod}\label{bfs_pseudokod}
G – graf\\
s – wierzchołek początkowy\\
e – wierzchołek końcowy\\
q – kolejka FIFO\\
p – tablica zawierająca wskazanie na poprzednika i-tego wierzchołka\\
d – tablica zawierająca odleglość i-tego wierzchołka od s\\
\begin{algorithmic}
\Function{BFS}{$G,s,e$}
	\For{wierzcholek $u$ w $G$} 
   	\State $p[u]\gets nieokreslony$
   \EndFor
   
   \State $p[s]\gets s$
   \State umiesc $s$ w $q$
   
   \While{sa elementy w $q$}
   	\State zdejmij wierzcholek u z q
   	\If {$u$ = $e$} \Return
   	\EndIf
   	\For{wierzcholek $v$ sasiadujacy z $u$}
   		\If {$p[v]$ = nieokreslony}
   			\State $p[v]\gets u$
   			\State umiesc $v$ w $q$
   		\EndIf
   \EndFor
   \EndWhile
\EndFunction
\Statex
\Function{odtworz\_sciezke}{p, aktualny}
	\State $sciezka\gets aktualny$
   
   \While{$aktualny$ rozny od $s$}
   	\State $aktualny\gets p[aktualny]$
   	\State dodaj $aktualny$ do $sciezka$
   \EndWhile
   \Return sciezka
\EndFunction
\end{algorithmic}
\vspace{5mm}
Funkcja odtwórz\_ścieżkę jest identyczna dla wszystkich algorytmów. Zwraca ona ścieżkę odwrotną, tj. od wierzchołka końcowego do wierzchołka początkowego.

\subsubsection{Złożoność czasowa}
Pesymistyczna – O(V+E)\\
gdzie V i E to liczności kolejno wierzchołków (pól białych) i krawędzi w grafie

\subsection{Dijkstra}
Algorytm wybiera następny wierzchołek do analizy posługując się kolejką priorytetową. Priorytetem kolejki jest aktualnie wyliczona odległość od wierzchołka źródłowego s.

\subsubsection{Pseudokod}
G – graf\\
s – wierzchołek początkowy\\
e – wierzchołek końcowy\\
q – kolejka priorytetowa\\
p – tablica zawierająca wskazanie na poprzednika i-tego wierzchołka\\
d – tablica zawierająca odleglość i-tego wierzchołka od s\\
\begin{algorithmic}
\Function{Dijkstra}{$G,s,e$}
	\For{wierzcholek $u$ w $G$} 
   	\State $d[u]\gets nieskonczonosc$
   	\State $p[u]\gets nieokreslony$
   \EndFor
   
   \State $d[s]\gets 0$
   \State $p[s]\gets s$
   \State umieść $s$ w $q$ z priorytetem 0
   
   \While{sa elementy w $q$}
   	\State zdejmij z $q$ wierzchołek $u$ o najmniejszej wartości $d$
   	\If {$u$ = $e$} \Return
   	\EndIf
   	\For{wierzcholek $v$ sasiadujacy z $u$}
   		\State $odleglosc\gets d[u]+1$
   		\If {$odleglosc < $d[v]}
   			\State $d[u]\gets odleglosc$
   			\State $p[v]\gets u$
   			\State zmień priorytet $v$ w $q$ na $odległosc$
   		\EndIf
   \EndFor
   \EndWhile
\EndFunction
\end{algorithmic}

\subsubsection{Złożoność czasowa}
Implementacja opiera się na kolejce priorytetowej opisanej w rozdziale \ref{priorityQueue}.
Pesymistyczna złożoność czasowa wynosi O(E * logV).

\subsection{A*}
Algorytm A* jest algorytmem heurystycznym. Działa podobnie jak alg. Dijkstry. Dla każdego
analizowanego wierzchołka wyznacza jednak nie tylko odległość od wierzcholka startowego, ale również
przewidywaną przez heurystykę odległość od wierzchołka końcowego. W ten sposób większy priorytet mają
(są w pierwszej kolejności wybierane) wierzchołki leżące bliżej celu.
Heurystyka dla analizowanego problemu liczy odległość pomiędzy zadanym wierzchołkiem, 
a wierzchołkiem początkowym. Z racji, że rozpatrywane są drogi na rastrze, to odległość ta 
liczona jest według metryki Manhattan.
\subsubsection{Pseudokod}
G – graf\\
s – wierzchołek początkowy\\
e – wierzchołek końcowy\\
q – kolejka priorytetowa\\
p – tablica zawierająca wskazanie na poprzednika i-tego wierzchołka\\
d – tablica zawierająca odleglość i-tego wierzchołka od s\\
f[i] = d[i] + heurystyka(i,e) – wartość priorytetu dla i-tego wierzchołka\\
\begin{algorithmic}
\Function{heurystyka}{$a,b$}
	\State \textbf{return} \(\lvert{a.x-b.x}\rvert-\lvert{a.y-b.y}\rvert\)
\EndFunction
\Statex
\Function{A\_star}{$G,s,e$}
	\For{wierzcholek $u$ w $G$} 
   	\State $d[u]\gets nieskonczonosc$
   	\State $p[u]\gets nieokreslony$
   \EndFor
   
   \State $d[s]\gets 0$
   \State $p[s]\gets s$
   \State umieść $s$ w $q$ z priorytetem 0
   
   \While{sa elementy w $q$}
   	\State zdejmij z $q$ wierzchołek $u$ o najmniejszej wartości $d$
   	\If {$u$ = $e$} \Return
   	\EndIf
   	\For{wierzcholek $v$ sasiadujacy z $u$}
   		\State $odleglosc\gets d[u]+1$
   		\If {$odleglosc < $d[v]}
   			\State $d[u]\gets odleglosc$
   			\State $p[v]\gets u$
   			\State $prio\gets odleglosc + \Call{heurystyka}{v,e}$
   			\State zmień priorytet $v$ w $q$ na $prio$
   		\EndIf
   \EndFor
   \EndWhile
\EndFunction
\end{algorithmic}
\subsubsection{Złożoność czasowa}
A* posiada najlepszą spośród wymienionych algorytmów pesymistyczną złożoność czasową wynoszącą O(E).

\section{Algorytm generujący dane}
Zgodnie z założeniami aplikacja umożliwia uruchomienie algorytmów na losowo wygenerowanym
rastrze. W tym celu zaimplementowano prosty algorytm opisany w wielu źródłach jako
przerzukiwanie DFS (ang. Depth-First Search) z nawrotami (ang. backtracking). Rozszerzono
go o możliwość rysowania rastra z zadaną ilością pól białych (wierzchołków grafu) i krawędzi
łączących te pola (krawędzi grafu). W ten sposób użytkownik może określić nie tylko wielkość
rastra, lecz również podać wymienione wyżej parametry opisujące graf, co umożliwia testowanie
złożoności zaimplementowanych algorytmów.
Początkowo raster wypełniony jest czarnymi polami. Tworzenie ścieżek rozpoczyna się od punktu
startowego, którego współrzędne to 0,0 (lewy górny róg rastra). Punkt ten staje się pierwszym 
polem białym. Punkt końcowy jest zainicjowany współrzędnymi pola startowego. Następnie algorytm
działa według poniższych kroków:\\

Dopóki raster zawiera pola czarne i nie osiągnięto zadanej liczby wierzchołków i krawędzi:
\begin{enumerate}
\item Wylosuj ilość pól k z przedziału 1-3.
\item Jeżeli z aktualnego pola można ruszyć się o k pól w dowolnym kierunku poruszając się wyłącznie po polach czarnych:
	\begin{enumerate}[label*=\arabic*.]
	\item Losowo wybierz kierunek, w którym można poruszyć się o k pól
	\item Jeśli krok o długości 1 w wylosowanym kierunku nie stworzy zbyt dużo nowych krawędzi:
		\begin{enumerate}[label*=\arabic*.]
		\item Zrób 1 krok w wylosowanym kierunku: zmień kolor pola na biały i dodaj pole do stosu
		\item Jeśli obie współrzędne aktualnego pola są większe od współrzędnych punktu końcowego:
				\begin{enumerate}[label*=\arabic*.]
				\item Ustaw aktualne pole jako pole końcowe
				\end{enumerate}
		\end{enumerate}
	\item Zwiększ licznik wierzchołków o 1 i licznik krawędzi o liczbę nowo utworzonych krawędzi
	\end{enumerate}
\item Zdejmij pole ze stosu
\item Ustaw je jako aktualne pole
\end{enumerate}
Po zakończeniu działania algorytmu odległość pomiędzy polami startowym i końcowym jest, w zależności od
przypadku, zmaksymalizowana lub temu bliska.

\section{Opis wykorzystanych struktur danych}
\subsection{Raster}
Raster reprezentowany jest w programie przez dwuwymiarową, dynamicznie alokowaną tablicę
wypełnioną liczbami całkowitymi. Ściana (czarne pole) reprezentowana jest przez wartość 0, 
natomiast droga (białe pole) ma wartość 1.

\subsection{Kolejka priorytetowa}\label{priorityQueue}
Algorytm BFS korzysta z prostej kolejki FIFO. Użyta została w tym celu szybka kolejka 
deque z biblioteki standardowej. Z kolei algorytmy Dijksta i A* wymagają bardziej złożonego
kontenera, w którym kolejność elementów określona jest za pomocą liczby - priorytetu.
W projekcie użyta została kolejka priority\_queue z
biblioteki standardowej. Na potrzeby algorytmu obudowano ją strukturą z pomocnicznymi
metodami. Kolejka ta oparta jest na kopcu binarnym przez co wstawianie do niej elementów
jest niezwykle szybkie. Jej wadą jest natomiast brak metody zmieniającej priorytet
elementu będącego już w kolejce. Brak takiej funkcjonalności jest prawdopodobnie
spowodowany niemałym kosztem wspomnianej operacji. Można jednak, co zostało zrealizowane w
aplikacji, zrezygnować ze zmiany priorytetu istniejącego elementu, a w zamian dodać
do kolejki ten sam element z nową wartością priorytetu. Kolejka może zawierać duplikaty,
co z pewnością ją spowalnia, jednak algorytmy na niej oparte działają poprawnie.

\subsection{Macierze poprzedników i odległości}
Każdy z algorytmów posługuje się macierzą poprzedników - dwuwymiarową, dynamicznie
alokowaną tablicą zawierającą
pary liczb całkowitych. Liczby te są indeksami w rastrze. Oznacza to, że pod i-tym wierszem
i j-tą kolumną w mecierzy poprzedników znajdują się współrzędne poprzednika pola o
współrzędnych i,j. Macierz ta jest aktualizowana podczas działania algorytmu szukającego
najkrótszej ścieżki. Dzięki temu po zakończeniu działania algorytmu możliwe jest
odtworzenie znalezionej ścieżki. Funkcja zwracająca znalezioną ścieżkę, korzystająca
przy tym z macierzy poprzedników została przedstawiona w rozdziale \ref{bfs_pseudokod}.\\
Algorytmy Dijkstra i A* posługują się dodatkowo macierzą odległości (dwuwymiarową tablicą
liczb całkowitych) zawierającą odległości punktu o współrzędnych i,j od punktu startowego. Początkowo
wszystkie elementy tablicy zawierają liczbę INT\_MAX.

\section{Pomiary czasów wykonania}
\subsection{Breadth-First Search}
\begin{longtable}{| m{1.7cm}| m{1.7cm} | m{1.7cm} | m{1.2cm} |}
\hline
V    &      E    &      t(n)[us] &  q(n)  \\ \hline
200  &      308  &      23.4  &     1.131 \\ \hline
400  &      663  &      43.8  &     1.012 \\ \hline
800  &      1338  &     90.2  &     1.036 \\ \hline
1600  &     2757  &     190.2  &    1.072 \\ \hline
3200  &     5725  &     368.2  &    1.013 \\ \hline
6400  &     11555  &    735.6  &    1.006 \\ \hline
12800  &    22744  &    1487.8  &   1.028 \\ \hline
25600  &    46417  &    2933.6  &   1.000 \\ \hline
51200  &    91238  &    5809.6  &   1.001 \\ \hline
102400  &   189028  &   11626.2  &  0.979 \\ \hline
204800  &   377151  &   24930.6  &  1.052 \\ \hline
409600  &   756946  &   49056.4  &  1.032 \\ \hline
819200  &   1520387  &  96786.8  &  1.016 \\ \hline
1638400  &  3000025  &  202555.2  & 1.072 \\ \hline
3276800  &  5988433  &  418533.6  & 1.109 \\ \hline
\end{longtable}

Wartość współczynika q w powyższej tabeli dla różnych wielkości problemu wynosi z dużą dokładością 1,
co świadczy o tym, że algorytm posiada złożoność zgodną z teoretyczną.

\subsection{Dijkstra}
\begin{longtable}{| m{1.7cm}| m{1.7cm} | m{1.7cm} | m{1.2cm} |}
\hline
V    &      E    &      t(n)[us] &  q(n)  \\ \hline
200  &      308  &      23.2  &     1.582 \\ \hline
400  &      663  &      56.4  &     1.580 \\ \hline
800  &      1338  &     113.8  &    1.416 \\ \hline
1600  &     2757  &     237.8  &    1.301 \\ \hline
3200  &     5725  &     508.6  &    1.225 \\ \hline
6400  &     11555  &    1011.6  &   1.112 \\ \hline
12800  &    22744  &    2046.0  &   1.059 \\ \hline
25600  &    46417  &    4233.0  &   1.000 \\ \hline
51200  &    91238  &    8856.4  &   0.996 \\ \hline
102400  &   189028  &   17460.8  &  0.891 \\ \hline
204800  &   377151  &   37164.0  &  0.897 \\ \hline
409600  &   756946  &   76418.8  &  0.870 \\ \hline
819200  &   1520387  &  158200.0  & 0.851 \\ \hline
1638400  &  3000025  &  328326.4  & 0.851 \\ \hline
3276800  &  5988433  &  685512.4  & 0.849 \\ \hline
\end{longtable}

Zaprezentowane w tabeli rezultaty mogę wskazywać, że w ocenie złożoności teoretycznej algorytmu Dijkstry
nastąpiło przeszacowanie (q malejące).
\newpage
\subsection{A*}
\begin{longtable}{| m{1.7cm}| m{1.7cm} | m{1.7cm} | m{1.2cm} |}
\hline
V    &      E    &      t(n)[us] &  q(n)  \\ \hline
200  &      311  &      24.6  &     0.963 \\ \hline
400  &      629  &      31.8  &     0.616 \\ \hline
800  &      1173  &     45.2  &     0.469 \\ \hline
1600  &     2775  &     184.6  &    0.810 \\ \hline
3200  &     5734  &     455.0  &    0.966 \\ \hline
6400  &     11439  &    822.0  &    0.875 \\ \hline
12800  &    22958  &    1804.2  &   0.957 \\ \hline
25600  &    46603  &    3827.6  &   1.000 \\ \hline
51200  &    92750  &    4801.4  &   0.630 \\ \hline
102400  &   188290  &   12680.8  &  0.820 \\ \hline
204800  &   378465  &   28074.4  &  0.903 \\ \hline
409600  &   753319  &   52120.0  &  0.842 \\ \hline
819200  &   1517679  &  124889.4  & 1.002 \\ \hline
1638400  &  3001474  &  218974.0  & 0.888 \\ \hline
3276800  &  5989141  &  601806.8  & 1.223 \\ \hline
\end{longtable}

Dla niektórych wielkości problemu (np. V=800, V=51200) współczynnik q jest znacznie mniejszy od 1.
Zmierzony dla tych przypadków czas jest wyraźnie krótszy od analogicznych czasów dla pozostałych algorytmów.
Jest to potwierdzeniem tego, że zastosowanie heurystyki daje w pewnych przypadkach wymierne korzyści.

\subsection{Uwagi}
Współczynniki q dla mniejszych rozmiarów problemów niż zaprezentowane w tabelach okazywały się kilkukrotnie
wyższe od jedności. Przyczyną tego zjawiska może być bardzo szybki czas wykonania funkcji odnajdujących najkrótsze
ścieżki i tym samym niemożność dokonania wiarygodnego pomiaru upłyniętego czasu (czasy rzędu kilku us).

\end{document}